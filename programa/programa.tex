\documentclass[letterpaper,10pt,onecolumn]{article}
\usepackage[spanish]{babel}
\usepackage[utf8]{inputenc}
\usepackage{amsfonts}
\usepackage{amsthm}
\usepackage{amsmath}
\usepackage{mathrsfs}

\usepackage{enumitem}
\usepackage[pdftex]{color,graphicx}
\usepackage{hyperref}
\usepackage{listings}
\usepackage{calligra}
\usepackage{url}
%\usepackage{algpseudocode} 
\DeclareMathAlphabet{\mathcalligra}{T1}{calligra}{m}{n}
\DeclareFontShape{T1}{calligra}{m}{n}{<->s*[2.2]callig15}{}
\newcommand{\scripty}[1]{\ensuremath{\mathcalligra{#1}}}
\lstloadlanguages{[5.2]Mathematica}
\setlength{\oddsidemargin}{0cm}
\setlength{\textwidth}{490pt}
\setlength{\topmargin}{-40pt}
\addtolength{\hoffset}{-0.3cm}
\addtolength{\textheight}{4cm}

\begin{document}
\begin{center}

\includegraphics[width=490pt]{header.png}\\[0.5cm]

\textsc{\LARGE M\'etodos Computacionales 1}\\[0.1cm]
%\large Nombre Profesora \\[0.5cm]

\end{center}

\large \noindent\textsc{Nombre del curso:}  M\'etodos Computacionales 1 % aqui %nombre del curso 
  
\noindent\textsc{C\'odigo del curso:}  FISI-2526 %Aqui el codigo del %curso 

\noindent\textsc{Unidad acad\'emica:} Departamento de F\'isica

\noindent\textsc{Prerrequisitos}: Introducci\'on a la Programaci\'on
(ISIS-1221). \'Algebra Lineal I (MATE-1105). 

\noindent\textsc{Cr\'editos}: 3

\noindent\textsc{Periodo acad\'emico:} 202020 %Aqui el periodo, %p.ej. 201510 

\noindent\textsc{Horario:} Mi y Vi, 17:30 a 18:45
%Aqui el horario, %p.ej. Ma y Ju, 10:00 a 11:20 

\noindent\rule{\textwidth}{1pt}\\[-0.3cm]

\normalsize \noindent\textsc{Nombre profesor magistral:}
Jaime Ernesto Forero Romero
%Aqui nombre del profesor principal   

\noindent\textsc{Correo electr\'onico:}
\href{mailto:je.forero@uniandes.edu.co}{\nolinkurl{je.forero@uniandes.edu.co}}
%Cambie address por su direccion de correo uniandes 

\noindent\textsc{Horario y lugar de atenci\'on:} 
con cita previa
% horario de atencion
\\[-0.1cm]


\normalsize \noindent\textsc{Nombre profesor Complementaria:}
John Fredy Su\'arez P\'erez
%Aqui nombre del profesor principal 

\noindent\textsc{Correo electr\'onico:}
\href{mailto:jf.suarez@uniandes.edu.co}{\nolinkurl{jf.suarez@uniandes.edu.co}}
%Cambie address por su direccion de correo uniandes 

\noindent\textsc{Horario de atenci\'on:} con cita previa. 
\\[-0.1cm]

\noindent\textsc{Repositorio del curso:} \url{https://github.com/ComputoCienciasUniandes/MetComp1_202020} 
%\href{mailto: jd.prada1760@uniandes.edu.co}{\nolinkurl{jd.prada1760@uniandes.edu.co}}

%Cambie address por direccion de correo uniandes del profesor
%complementario 

%\noindent\textsc{Horario y lugar de atenci\'on:} %Aqui horario y
%lugar de atencion del profesor complementario, p.ej. Vi, 15:00 a
%17:00, Oficina Ip102 
%\\[-0.1cm]
%Repetir esto en caso de varios profesores complementarios

%\noindent\textsc{Nombre monitor(a):} %Aqui nombre del monitor si aplica

%\noindent\textsc{Correo electr\'onico:}
%\href{mailto:address@uniandes.edu.co}{\nolinkurl{address@uniandes.edu.co}}
%%Cambie address por direccion de correo uniandes del monitor 

%\noindent\textsc{Horario y lugar de atenci\'on:} %Aqui horario y
%lugar de atencion del monitor, p.ej. Vi, 15:00 a 17:00, Oficina Ip102 

\noindent\rule{\textwidth}{1pt}\\[-0.1cm]

\newcounter{mysection}
\addtocounter{mysection}{1}

\noindent\textbf{\large \Roman{mysection} \quad Introducci\'on}\\[-0.2cm]

%Este espacio es para hacer una introduccion al curso, evidenciando la
%propuesta metodologica. Debe ser clara y precisa. 

\noindent\normalsize Los m\'etodos computacionales son un aspecto
inseparable de cualquier \'area de trabajo en ciencia e ingenier\'ia.
Esto se debe a la facilidad de acceso a computadoras programables  y
su aumento exponencial en capacidad de procesamiento.  
Estos recursos para el c\'omputo s\'olo se puede aprovechar si las
personas interesadas son capaces de utilizarlos tecnolog\'{\i}a
de manera eficiente.
De manera complementaria, la obtenci\'on y comprensi\'on de los
resultados obtenidos  con estos m\'etodos computacionales requieren
una comprensi\'on b\'asica de probabilidad y estad\'istica. \\[0.1cm]  

\stepcounter{mysection}
\noindent\textbf{\large \Roman{mysection} \quad Objetivos}\\[-0.2cm]

%En este espacio se debe precisar el ente visor del curso y el
%proposito ideal al finalizar el curso. 
\noindent\normalsize En el curso se presentan
algoritmos y t\'ecnicas computacionales b\'asicas para:

\begin{itemize}
\item analizar datos y modelos con conceptos probabil\'isticos y
  m\'etodos estad\'isticos \\[-0.6cm] 
\item resolver num\'ericamente problemas que involucren derivadas,
  integrales y sistemas de ecuaciones algebraicas \\[-0.6cm]
\end{itemize} 
\vspace*{0.5cm} 

\stepcounter{mysection}
\noindent\textbf{\large \Roman{mysection} \quad Competencias a
  desarrollar}\\[-0.2cm] 

%En este espacio se describen las habilidades que el estudiante desarrollara en el transcurso del curso.

\noindent\normalsize Al finalizar el curso, se espera que el
estudiante adquiera las siguientes habilidades. 

\begin{itemize}
\item Analizar datos usando apropiadamente conceptos
  b\'asicos de probabilidad y estad\'istica en la soluci\'on de
  problemas en ciencias y otras \'areas: variables aleatorias, valores
  esperados, m\'etodos de Monte-Carlo, pruebas de hip\'otesis: 
 \begin{itemize}
  \item Diferenciar distribuciones de probabilidad comunmente
    encontradas en ciencias: normal, poisson, log-normal, etc. 
  \item Hacer histogramas para identificar la distribuci\'on de los
    datos como funci\'on de una variable espec\'ifica. 
  \item Estimar la moda, media, mediana, desviaci\'on estandar,
    curtosis y otros momentos de series de datos. 
  \item Realizar pruebas cuantitativas para ver si dos series de datos
    corresponden a la misma distribuci\'on de probabilidad. 
  \item Utilizar datos para proponer modelos param\'etricos y acotar
    los terminos libres de este.  
  \item Solucionar sistemas de ecuaciones algebraicas lineales.  
  \item Resolver problemas de m\'inimos cuadrados, planteados desde el
    sistema de ecuaciones lineales correspondiente. 
  \item Realizar reducci\'on de dimensionalidad de datos a trav\'es
    del m\'etodo de componentes principales. 
 \end{itemize}
\item Usar herramientas de Python para an\'alisis estad\'istico de
  datos, para la soluci\'on de problemas en ciencias y otras
  \'areas.\\[-0.6cm] 
\begin{itemize}
   \item Implementar computacionalmente modelos f\'isico--mat\'ematicos
     para el an\'alisis de datos utilizando \texttt{clases} y
     \texttt{funciones}.\\[-0.5cm] 
    \item Utilizar de la biblioteca \texttt{pandas} para el manejo eficiente de los datos
    teniendo en cuenta los siguientes procedimientos: preprocesamiento, re-formato de
    columnas y normalizaci\'on.\\[-0.5cm] 
   \item Utilizar la biblioteca \texttt{numpy} en la implementaci\'on
     de los algoritmos matem\'aticos que representen los diferentes
     sistemas en estudio.\\[-0.5cm] 
    \item  Utilizar la biblioteca \texttt{matplotlib} para visualizar
      datos.\\[-0.5cm] 
\end{itemize}

\item Solucionar numéricamente problemas sencillos de cálculo
  diferencial, cálculo integral y álgebra lineal: cálculo de
  integrales, cálculo de derivadas, sistemas de ecuaciones lineales,
  integraci\'on de funciones, problemas de valores propios. \\[-0.6cm]   

\begin{itemize}
    \item Calcula num\'ericamente la primera y segunda derivada de
      funciones de una variable. 
    \item Calcula num\'ericamente integrales unidimensionales
      definidas e indefinidas con los m\'etodos siguientes: trapecio,
      y cuadratura gaussiana. 
    \item Encuentra num\'ericamente la soluci\'on a sistemas de
      ecuaciones algebraicos lineales. 
    \item Utiliza bibliotecas de \'algebra lineal para calcular
      valores y vectores propios. 
    \item Interpreta los vectores y valores propios de una matriz de
      covarianza como herramientas para la reducci\'on de
      dimensionalidad de un conjunto de datos. 
\end{itemize}
\end{itemize}
%de manera aut\'onoma la soluci\'on num\'erica a una variedad de fen\'omenos representados mediante ecuaciones diferenciales o distribuciones probabil\'{\i}sticas.

\vspace*{0.5cm} 

\stepcounter{mysection}
\noindent\textbf{\large \Roman{mysection} \quad Contenido por
  semanas}\\[-0.2cm]  

%Se expone de forma ordenada toda la tematica a tratar del curso. Debe planearse para 15 semanas.

\noindent\textbf{\textsc{Semana 1}}\\[-0.5cm]
\begin{itemize}
\item Temas: 
Presentaci\'on del curso. Uso de Python en la nube. Comandos b\'asicos Unix. 
Repaso de Python: variables, listas, iteraci\'on diccionarios, lectura
de archivos, condicionales.\\[-0.6cm] 
%\item Lecturas preparatorias: 
\end{itemize}


\noindent\textbf{\textsc{Semana 2}} \\[-0.5cm]
\begin{itemize}
\item Temas: 
Repaso de Python: funciones, clases, objetos, numpy, matplotlib. \\[-0.6cm] 
%\item Lecturas preparatorias: 
\end{itemize}

\noindent\textbf{\textsc{Semana 3}}\\[-0.5cm]
\begin{itemize}
\item Temas:  
Introducci\'on a la estad\'istica. Estad\'istica descriptiva.
\\[-0.6cm]  
%\item Lecturas preparatorias: \\[-0.6cm] 
\end{itemize}

\noindent\textbf{\textsc{Semana 4}}\\[-0.5cm]
\begin{itemize}
\item Temas:  Elementos b\'asicos de probabilidad. 
\\[-0.6cm] 
%\item Lecturas preparatorias: \\[-0.6cm]    
\end{itemize}

\noindent\textbf{\textsc{Semana 5}}\\[-0.5cm]
\begin{itemize}
\item Temas: Variables aleatorias y valores esperados.
  \\[-0.6cm] 
%\item Lecturas preparatorias: \\[-0.6cm]
\end{itemize}


\noindent\textbf{\textsc{Semana 6}}\\[-0.5cm]
\begin{itemize}
\item Temas:  Variables aleatorias especiales. \\[-0.6cm]  
%\item Lecturas preparatorias: \\[-0.6cm]  
\end{itemize}


\noindent\textbf{\textsc{Semana 7}}\\[-0.5cm]
\begin{itemize}
\item Temas: Distribuciones de estad\'isticas de sampleo.
\\[-0.6cm] 
%\item Lecturas preparatorias:  \\[-0.6cm] 
\end{itemize}

\noindent\textbf{\textsc{Semana 8}}\\[-0.5cm]
\begin{itemize}
\item Temas: Estimaci\'on de par\'ametros.
\\[-0.6cm] 
%\item Lecturas preparatorias: \\[-0.6cm] 
\end{itemize}

\noindent\textbf{\textsc{Semana 9}}\\[-0.5cm]
\begin{itemize}
\item Temas: Prueba de hip\'otesis.\\[-0.6cm]   
%\item Lecturas preparatorias:\\[-0.6cm]
\end{itemize}


\noindent\textbf{\textsc{Semana 10}}\\[-0.5cm]
\begin{itemize}
\item Temas: Regresi\'on.\\[-0.6cm]
\end{itemize}

\noindent\textbf{\textsc{Semana 11}}\\[-0.5cm]
\begin{itemize}
\item Temas: Integraci\'on. \\[-0.6cm]
%\item Lecturas preparatorias: Cap\'itulo 6 (Integration) del libro de
%Landau.\\[-0.6cm]
\end{itemize}

\noindent\textbf{\textsc{Semana 12}}\\[-0.5cm]
\begin{itemize}
\item Temas: Derivadas. Ra\'ices de ecuaciones. \\[-0.6cm]
%\item Lecturas preparatorias: Cap\'itulos 7.I (Numerical
%  Differentiation) y 7.II (Trial-and-Error Searching) del libro de
%  Landau.\\[-0.6cm] 
\end{itemize}

\noindent\textbf{\textsc{Semana 13}}\\[-0.5cm]
\begin{itemize}
\item Temas: Interpolación. \\[-0.6cm]
%\item Lecturas: \\[-0.6cm] 
\end{itemize}

\noindent\textbf{\textsc{Semana 14}}\\[-0.5cm]
\begin{itemize}
\item Temas: Soluci\'on de sistemas de ecuaciones lineales. 
\\[-0.6cm] 
%\item Lecturas preparatorias:  Cap\'itulo 8 (Matrix Equation Solutions)
%  del libro de Landau. \\[-0.6cm]
\end{itemize}


\noindent\textbf{\textsc{Semana 15}}\\[-0.5cm]
\begin{itemize}
\item Temas: Ajustes por m\'inimos cuadrados con m\'etodos matriciales. 
%\item Lecturas preparatorias: Secciones 6.3.1 (Principal Component
%  Regression) y 10.2 (Principal Component Analysis) del libro ISL.\\[-0.6cm]
\end{itemize}

\noindent\textbf{\textsc{Semana 16}}\\[-0.5cm]
\begin{itemize}
\item Temas: Autovalores y autovectores. An\'alisis de Componentes Principales. \\[-0.6cm]
\end{itemize}


\vspace*{0.5cm} 
\stepcounter{mysection}
\noindent\textbf{\large \Roman{mysection} \quad
  Metodolog\'ia}\\[-0.2cm] 

Los $3$ cr\'editos del curso corresponden a $9$ horas de dedicaci\'on semanal.
Cada semana habr\'a $3.75$ horas sincr\'onicas distribu\'idas en $2\times 1.25=2.5$ horas de magistral 
y $1.25$ horas de complementaria. 
Las $5.25$ horas restantes corresponden a trabajo individual para la
preparaci\'on de los contenidos de la semana.
De esto sugerimos dedicarle $4$ horas a la magistral y $1.25$ horas a
la complementaria.

\begin{itemize}
    \item Cada semana se publica un m\'odulo nuevo en BrightSpace (curso MÉTOD.COMPUT.1(REFORMA 202020)). 
      Las actividades del m\'odulo se deben trabajar antes de la clase magistral. 
      El tiempo de estudio y trabajo de este material no debe exceder
      el tiempo previsto para trabajo individual. 
    \item Durante las clases sincr\'onicas de la magistral habr\'a una
      sesi\'on de zoom de $30$ minutos para presentar la parte
      te\'orical del tema del d\'ia, mostrar un ejemplo de implementaci\'on
      y resolver las dudas que se tengan sobre el tema
      del d\'ia.
      Las dudas ser\'an recogidas en el 
      \textbf{\href{https://uniandes.padlet.org/jeforero/dddxuuqa7q4cadgs}{Padlet
        de la magistral}}.
      Los siguientes $45$ minutos se dedicar\'an a la resoluci\'on de
      un ejercicio que se debe responder a trav\'es de BrightSpace. La
      interacci\'on se har\'a a trav\'es de slack. 
    \item
      Durante las clases sincr\'onicas de la complementaria habr\'a
      una sesi\'on de Clase Remota en SICUA de
      $30$ minutos para resolver dudas de la clase magistral de la
      semana anterior.
      Las dudas ser\'an recogidas en el
      \textbf{\href{https://uniandes.padlet.org/jfsuarez/vj0aroqbr0kw9fmq}{Padlet
        de la complementaria}}.
      Los siguientes $45$ minutos se dedicar\'an a la resoluci\'on de
      un ejercicio que se deber\'a responder a trav\'es de SICUA. 
      La interacci\'on se dar\'a a trav\'es de slack. 
\end{itemize}

Para el desarrollo de la clase se usará el ambiente Binder accesible
desde el
\href{https://github.com/ComputoCienciasUniandes/MetComp1_202020}{repositorio
  del curso}. 

%Se describen las tecnicas y metodos para el desarrollo exitoso del curso.

%\noindent\normalsize 


\vspace*{0.5cm} 
\stepcounter{mysection}
\noindent\textbf{\large \Roman{mysection} \quad Criterios de
  evaluaci\'on}\\[-0.2cm] 

En cada clase de magistral y complementaria habr\'a un ejercicio para
entregar. 
Solamente recibir\'an calificacion 16  de esos
ejercicios, 8 de la magistral y 8 de la complementaria.

El profesor de la clase magistral anunciar\'a los ejercicios que
recibien calificaci\'on cada dos semanas.
Al publicar el enunciado de un ejercicio no se sabe si ser\'a
calificado o no.

La nota num\'erica se calcula a partir de estas entregas de ejercicios de
programaci\'on. Los porcentajes se dividen en

\begin{itemize}
    \item Ocho (8) Ejercicios de programaci\'on propuestos en la
      Magistral: $70\%$.   
    \item Ocho (8) Ejercicios de programaci\'on propuestos en la
      Complementaria: $30\%$.  
\end{itemize}


Cada ejercicio recibe el siguiente porcentaje de calificaci\'on:
\begin{itemize}
\item Entrega de c\'odigo fuente que incluye las funciones, algoritmos
  y procedimientos que se solicitan en el enunciado: $30\%$.  
\item El c\'odigo corre sin errores dentro del int\'erprete de python: $30\%$.
\item El c\'odigo produce la respuesta correcta: $40\%$.
\end{itemize}
El c\'odigo se calificar\'a dentro del int\'erprete de binder del
repositorio del curso.
La nota definitiva se reporta con dos cifras decimales.

Las entregas de ejercicios que se publican durante la sesi\'on
sincr\'onica de magistral y complementaria se deben entregan en el horario de clase. 


Todas las entregas de talleres y ejercicios se har\'an a trav\'es de 
la plataforma habilitada (BrightSpace para la Magistral y SICUA para
la Complementaria)
{\bf No se aceptar\'a ninguna tarea por fuera de esas plataformas}, a
menos que ocurra un una falla en los servidores que afecte a {\bf todos} los
estudiantes del curso. 
No se toman en cuenta entregas marcadas como {\bf tarde} por la plataforma. 

Todas las entregas son \textbf{individuales}.  
El trabajo no individual incluye la colaboraci\'on con
personas no inscritas en el curso, i.e. a 
trav\'es de ``monitor\'ias''.
\textbf{Las \'unicas fuentes autorizadas} para reutilizaci\'on de c\'odigo son:
el repositorio del curso (
\url{https://github.com/ComputoCienciasUniandes/MetComp1_202020}) y
los libros de la bibliograf\'ia principal.
En casos de copia, trabajo no individual o uso de fuentes no
autorizadas en los ejercicios se llevar\'a el caso a comit\'e
disciplinario y la nota del curso queda como Pendiente Disciplinario
hasta que el comit\'e tome alguna decisi\'on. 




\vspace*{0.5cm} 

\stepcounter{mysection}
\noindent\textbf{\large \Roman{mysection} \quad
  Bibliograf\'ia}\\[-0.2cm] 

%Indicar los libros y la documentacion guia.


\noindent\normalsize Bibliograf\'ia principal:

\begin{itemize}

\item \textit{Introduction to Probability and Statistics for Engineers and
  Scientists}. Sheldon M. Ross. Third Edition. Elsevier Academic
  Press. 2004.

\item
\textit{A survey of Computational Physics - Enlarged Python Book}
. R. H. Landau, M. J. P\'aez, C. C. Bordeianu. WILEY. 2012.
\url{https://psrc.aapt.org/items/detail.cfm?ID=11578}




\end{itemize}

\noindent\normalsize Bibliograf\'ia secundaria:

\begin{itemize}

\item 
\textit{A student's guide to numerical methods},
I. H. Hutchinson. Cambdrige University Press. 2015.
\item
\textit{Introduction to statistics with python}.
Thomas Haslwanter.  Springer. 2016.

\item 
\textit{Pattern recognition and Machine Learning}.
Christopher Bishop. Springer. 2011.

\item
\textit{Data Analysis: A Bayesian Tutorial.} D. S. Sivia,
J. Skilling. Second Edition, Oxford Science Publications. 2006.

\item 
\textit{Statistical Mechanics: Algorithms and Computations.}
W. Krauth, Oxford Univ. Press. 

\item
\textit{An Introduction to Statistical Learning.} G. James, D. Witten,
T. Hastie, R. Tibshirani,
Springer. \url{http://www-bcf.usc.edu/~gareth/ISL/} 


\item
\textit{Elements of Scientific Computing}
Tveito A., Langtangen H.P., Nielsen B.F., Cai X. Spinger. 2010.

\item 
\textit{Introduction to Computation and Programming Using Python},
Guttag, J. V. The MIT Press. 2013.

\end{itemize}


\end{document}
